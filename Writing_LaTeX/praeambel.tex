

%%Festlegen von Metainformationen des Dokuments%%%
\title{\LaTeX\xspace für Politikwissenschaftler}
\subtitle{Anwendungsbeispiele}
\author{Robert Stelzle}




%%1.5facher Zeilenabstand, außer Fußnoten
\usepackage[onehalfspacing]{setspace}   %%Siehe KomaScript Dokumentation S. 40			

%% Blockzitate mit einfachem Zeilenabstand
\usepackage{etoolbox}
\BeforeBeginEnvironment{quote}{\par\begin{singlespace*}}
\AfterEndEnvironment{quote}{\end{singlespace*}\par}


%%%Schriften Laden%%%
\usepackage{amsmath}%VOR unicode-math Laden
\usepackage{fontspec}% Zum Laden von Schriften
\usepackage{libertinus}% Macht gesamte Schrifteneinstellung, einschl. Mathe
\usepackage{unicode-math} %Mathematischer Zeichensatz
\setmonofont[Scale=MatchLowercase,
	FakeStretch=0.9]{Consolas}   %% die sieht besser aus als die monotype von libertine
\usepackage{nicefrac}


%%%Seitenlayout und Textsatz(Worttrennung etc.)%%%
\usepackage{microtype} %Mikrotypografie; schönerer Blocksatz
\usepackage{xspace} %Leerzeichen Schützen

\usepackage{geometry} %Alles rund um Ränder
\geometry{lmargin=2.5cm, rmargin=3.5cm, tmargin=2.5cm, bmargin=2.5cm,
	includeheadfoot,} %Kopf- und Fußzeile auch innerhalb der margins

%Kopf/Fußzeilen uvm.
\usepackage[automark,
headsepline=1pt, footsepline=1pt,
]{scrlayer-scrpage} 
\ohead{Robert Stelzle}  %%Kopfzeile außen (hier rechts, weil einseitiges 
					%%Dokument)
\ofoot*{Abgabedatum} %*
\ihead{\headmark}
\setkomafont{pagenumber}{\itshape}
\ifoot*{Seite \pagemark}
\chead{}
\cfoot*{}


%%%Sprachenspezifisches%
\usepackage{babel}%     Alles sprachenspezische wie Datum, Titel, usw
\usepackage{csquotes}% 	Schickere (bzw. richtige) Anführungszeichen mit \enquote{}



%%Literatur

\usepackage[backend=biber, 
style=authoryear-ibid,minnames=1,maxnames=3,isbn=false,date=year, sorting=nyt]{biblatex}% 
%BIbliografie
\addbibresource{literatur.bib}   %%% Hier ggf. die korrekte .bib Datei einbinden
\DefineBibliographyStrings{ngerman}{andothers={et\,al\adddot}}  %% et. al definieren
\DefineBibliographyStrings{ngerman}{nodate={o.\,D.}} 			%% o.D. Definieren





%%Grafiken

\usepackage{graphicx}% Alles rund um Grafik


%%Tabellen

\usepackage{xcolor}%Für Farbe in Tabellen usw.
\usepackage{booktabs} %Schönere Tabellen
\usepackage{array}   %Erweiterte Spaltentypen
\usepackage{tabularx}


%%Sonstiges

\usepackage{blindtext}  %% Mit \blindtext Beispieltext (statt Lorem Ipsum) einfügen
\usepackage{url} %URLs in Verbatime und URL Freundlicher Zeilenumbruch + verlinkung
\usepackage{varioref} %Schönere Referenzen
\usepackage{hyperref}
\usepackage{cleveref}
\usepackage{capt-of}


\setkomafont{captionlabel}{\bfseries\small\sffamily} %Label von Gleitobjekten Fett & Serifenlos
\setkomafont{caption}{\small\sffamily} %Über-/Unterschriften von Gleitobjekten Serifenlos




%%----Eigene Definitionen---

\newcommand\oAE {o.\,Ä.\xspace}
\newcommand\zB {z.\,B.\xspace}


%% Schicke Autorennennung unter Blockzitat. Je nachdem ob das noch in die Zeile passt oder nicht, aber immer rechtsbündig
\newcommand*{\signed}[1]%
	{\unskip\hspace*{1em plus 1fill}%
	\nolinebreak[3]\hspace*{\fill}\mbox{\emph{#1}}}



%Anpassung Abstände um subsection
\RedeclareSectionCommand[%
beforeskip=-1.5\lineskip,
afterskip=\lineskip
]{subsection}

