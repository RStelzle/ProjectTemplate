\documentclass[
  fontsize=11pt,
  parskip=half-, %zusätzliche halbe Zeile Abstand vor neuem Absatz
  english,ngerman,
  paper=a4,
  %twoside, %Wenn drinn, guckt er, dass ungerade Seiten links am Rand und
  			%gerade Seiten rechts am  Rand sind. Wenn man halt nicht nur ne PD
  			%sondern n Buch oder so machen will
  usegeometry, %Infos an geometry weiterreichen
  toc=listof,
  captions=tableabove,
  %showframe,
  bibliography=totoc, %mit totoc ohne Nummer ins TOC mit nottotoc nicht im TOC
]{scrartcl}




%%Festlegen von Metainformationen des Dokuments%%%
\title{\LaTeX\xspace für Politikwissenschaftler}
\subtitle{Anwendungsbeispiele}
\author{Robert Stelzle}




%%1.5facher Zeilenabstand, außer Fußnoten
\usepackage[onehalfspacing]{setspace}   %%Siehe KomaScript Dokumentation S. 40			

%% Blockzitate mit einfachem Zeilenabstand
\usepackage{etoolbox}
\BeforeBeginEnvironment{quote}{\par\begin{singlespace*}}
\AfterEndEnvironment{quote}{\end{singlespace*}\par}


%%%Schriften Laden%%%
\usepackage{amsmath}%VOR unicode-math Laden
\usepackage{fontspec}% Zum Laden von Schriften
\usepackage{libertinus}% Macht gesamte Schrifteneinstellung, einschl. Mathe
\usepackage{unicode-math} %Mathematischer Zeichensatz
\setmonofont[Scale=MatchLowercase,
	FakeStretch=0.9]{Consolas}   %% die sieht besser aus als die monotype von libertine
\usepackage{nicefrac}


%%%Seitenlayout und Textsatz(Worttrennung etc.)%%%
\usepackage{microtype} %Mikrotypografie; schönerer Blocksatz
\usepackage{xspace} %Leerzeichen Schützen

\usepackage{geometry} %Alles rund um Ränder
\geometry{lmargin=2.5cm, rmargin=3.5cm, tmargin=2.5cm, bmargin=2.5cm,
	includeheadfoot,} %Kopf- und Fußzeile auch innerhalb der margins

%Kopf/Fußzeilen uvm.
\usepackage[automark,
headsepline=1pt, footsepline=1pt,
]{scrlayer-scrpage} 
\ohead{Robert Stelzle}  %%Kopfzeile außen (hier rechts, weil einseitiges 
					%%Dokument)
\ofoot*{Abgabedatum} %*
\ihead{\headmark}
\setkomafont{pagenumber}{\itshape}
\ifoot*{Seite \pagemark}
\chead{}
\cfoot*{}


%%%Sprachenspezifisches%
\usepackage{babel}%     Alles sprachenspezische wie Datum, Titel, usw
\usepackage{csquotes}% 	Schickere (bzw. richtige) Anführungszeichen mit \enquote{}



%%Literatur

\usepackage[backend=biber, 
style=authoryear-ibid,minnames=1,maxnames=3,isbn=false,date=year, sorting=nyt]{biblatex}% 
%BIbliografie
\addbibresource{literatur.bib}   %%% Hier ggf. die korrekte .bib Datei einbinden
\DefineBibliographyStrings{ngerman}{andothers={et\,al\adddot}}  %% et. al definieren
\DefineBibliographyStrings{ngerman}{nodate={o.\,D.}} 			%% o.D. Definieren





%%Grafiken

\usepackage{graphicx}% Alles rund um Grafik


%%Tabellen

\usepackage{xcolor}%Für Farbe in Tabellen usw.
\usepackage{booktabs} %Schönere Tabellen
\usepackage{array}   %Erweiterte Spaltentypen
\usepackage{tabularx}


%%Sonstiges

\usepackage{blindtext}  %% Mit \blindtext Beispieltext (statt Lorem Ipsum) einfügen
\usepackage{url} %URLs in Verbatime und URL Freundlicher Zeilenumbruch + verlinkung
\usepackage{varioref} %Schönere Referenzen
\usepackage{hyperref}
\usepackage{cleveref}
\usepackage{capt-of}


\setkomafont{captionlabel}{\bfseries\small\sffamily} %Label von Gleitobjekten Fett & Serifenlos
\setkomafont{caption}{\small\sffamily} %Über-/Unterschriften von Gleitobjekten Serifenlos




%%----Eigene Definitionen---

\newcommand\oAE {o.\,Ä.\xspace}
\newcommand\zB {z.\,B.\xspace}


%% Schicke Autorennennung unter Blockzitat. Je nachdem ob das noch in die Zeile passt oder nicht, aber immer rechtsbündig
\newcommand*{\signed}[1]%
	{\unskip\hspace*{1em plus 1fill}%
	\nolinebreak[3]\hspace*{\fill}\mbox{\emph{#1}}}



%Anpassung Abstände um subsection
\RedeclareSectionCommand[%
beforeskip=-1.5\lineskip,
afterskip=\lineskip
]{subsection}



\begin{document}
	
	
%%Titelseite erstellen
%Die Titelseite ist ein wenig inspiriert von der Beispieltitelseite aus der LV
\newgeometry{lmargin=2cm, rmargin=2cm, tmargin=1cm, bmargin=2.5cm}
\begin{titlepage}
	\begin{center}
		\sffamily
		\vspace*{1.5cm}
		\Huge\bfseries\scshape
		Titel des Projekts\\
		\Large
		Untertitel des Projekts\\[3cm]
		\vspace*{0.2cm}
		\normalsize\mdseries\upshape
		Hausarbeit im Modul\\
		\large (XX) Beispielmodul\\
		\normalsize\mdseries\upshape
		in der Lehrveranstaltung\\
		\large Beispiellehrveranstaltung\\
		\normalsize bei Max Mustermann\\
		\large Abgabe ABGABEDATUM
		
		\vfill
		\upshape\Large
		Robert Stelzle\\
		\mdseries
		\url{r.stelzle@fu-berlin.de}\\[1ex]% Höhe vom kleinen x
		\normalsize
		Matrikelnummer: 1234567\\[1cm]
		BA Politikwissenschaft (150 LP)\\
		Allgemeine Berufsvorbereitung (30 LP)\\
		X. Fachsemester\\[1cm]
		\begin{minipage}[c]{0.65\linewidth}
			\centering Es gilt die Studien- und Prüfungsordnung 2016 des Fachbereichs Politik- und 
			Sozialwissenschaften der Freien Universität Berlin für den Bachelorstudiengang 
			Politikwissenschaft. 
		\end{minipage}
		
		
	\end{center}
\end{titlepage}
\restoregeometry
\setcounter{page}{1}


\tableofcontents
\newpage


\addsec{Abstract} \label{00abstract}
\vspace{-0.5cm}
\begin{quote}
	Everyone agrees that this issue is really important. But we do not
	know much about this specific question, although it matters a great
	deal, for these reasons. We approach the problem from this
	perspective. Our research design focuses on these cases and relies
	on these data, which we analyze using this method. Results show what
	we have learned about the question. They have these broader
	implications.
	
\end{quote}

\section{Einleitung}

Dies ist die Einleitung der folgenden Arbeit. Das Seitenlayout, etc. sind den Regularien der \textcite{arbeitsstellefurpolitischesoziologiederbundesrepublikdeutschlandamottosuhrinstitutderfreienuniversitatberlinHandreichungVerfassenHaus2019} nachempfunden.

\blindtext






%\listoftables

%\listoffigures
\newpage
\clearpage
\singlespacing
\printbibliography




\end{document}



